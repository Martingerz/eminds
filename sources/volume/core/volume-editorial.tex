\chapter*{Editorial}
We are really happy to participate in the creation of the new issue of the international journal on human-computer interaction that you have in your hands. Published by the Human Communication and Interaction Research Group (\url{www.hci-rg.com}), this journal tries to provide an forum for the exchange and development of high quality research work in different areas of human-computer interaction.


As it also happened in the last published issues, we have received a wide range of high quality research contributions this time, which made the paper selection process a really difficulty task. During this process, we not only tried to achieve the highest quality level possible, but also to provide our readers with state of the art research works covering as much different areas as possible in the scope of Human-Computer interaction. This includes topics like Usability Engineering, Mobile Computing, Augmented and Virtual Reality or Brain-Computer Interfaces, among many others.

In this issue, you will find outstanding research work in the following areas:

\begin{itemize}
	\item A Fuzzy Driven High Availability Cosynthesis Scheme:  a fuzzy-rule based engine that evaluates the user's availability requirements and weigh up the accuracy in the results to be able to reach the user expectations on the reliability \& availability by the system.

	\item Interface Design Patterns, Languages and Models: showing a graphical user interface (GUI) for \textit{FastComp}, a computational tool for the determination of the forces that a composite bolted joint subjected to multiaxial loads supports until failure occurs.

	\item Brain-Computer Interfaces Based on Attention and Complex Mental Tasks: Investigates two main applications of BCI systems in assistive technology: regain the movements or communications for people with motor disability and neuro feedback for training the subject to emit a specific brain activity.

	\item Grasp Strategy when Experiencing Hands of Various Sizes: Investigates the grasp strategy for different hand and object sizes, proposing a hand adaptation system by adjusting the optical zoom and actual object size.

	\item Intelligent-Miner: The Conceptual and Architectural Design of Web Based Data Mining Service: Presents the design for a Web enabled service that supports the implementation of data mining techniques, proposing a scalable, flexible and extensible data representation format.

\end{itemize}

In this issue you will also find a new section called \textsl{Usability Reports}, including a collection of design guidelines and heuristics developed by the Human Communication and Research Group to help you to improve the usability and accessibility of web-based documents. In this issue you will find the first two Usability Reports, describing the following design issues:

\begin{enumerate}

   \item Internationalization. Analyzing how to use the different interaction mechanisms required to make your site accessible to different cultures and languages.
   \item Web Shortcuts. Describes when, how, and how many Access Keys should be include in your designs in order to make then accessible.
\end{enumerate}

Now we must say goodbye until the next issue, but before doing so, we would like to acknowledge all members of our scientific panel of reviewers who really make the publication of this journal possible. We would also like to acknowledge all the members of the Human Communication and Interaction Research Group too for their participation, collaboration and personal involvement in the development of this project. 

\vspace*{37mm}

\begin{flushright}
The  e-Minds Editorial Team
\end{flushright}

\bigskip

%\begin{figure}[th]
%	\begin{center}
%		\includegraphics[width=0.25\textwidth]{Logos/hci-logo}
%	\end{center}
%\end{figure}

